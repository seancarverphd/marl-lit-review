\documentclass{article}
\title{Informal Review of Multi-Agent Reinforcement Learning Literature}
\author{Sean Carver @ Data Machines Corporation}
\begin{document}
\maketitle

\section{May 6, 2021}
\subsection{Positioning COMMS and JAMMERS to Defeat Each Other}
Quite by accident, I discovered a body of literature that I deem very
relevant to my COMMS and JAMMERS problem.

This work has an intersection with Multi-Agent Systems in that all the
drones are autonomous systems moving to achieve some goal under
partial information.  It seems that the authors I have read so far do
not specifically use reinforcement learning in their solution but
their solution begs to be extended to an RL framework.  (That said,
see \cite{testi2020reinforcement} for a non-adversarial MARL-example.)

Also, as far as I know, so far, no one [I am not certain of this,
yet], specifically talks about the problem I am interested in with
COMMS and JAMMERS---at least not in the way I have framed it---but the
relevance should be clear.

The problem they discuss is how to control a sensor network (in my
case, move the UAVs) in order to bring in the most information from
the sensors.  They use information-theoretic criteria as their
objective functions.  Beyond information theoretic criteria there are
other objectives that do similar things.

I'll list 2 scantly cited papers \cite{nicholson2007information,
  guerra2018collaborative}.  One more cited but less recent paper is
\cite{grocholsky2003scalable}.  There are certainly more papers
available on Google Scholar, but I haven't yet dug into the literature
yet.  The first paper (2007) was found in \cite{pechoucek2008defense},
a bound collection of papers and not available otherwise.  The second
is a more recent IEEE paper (2018) that is not available for free, and
I have only seen the abstract---but looks even more relevant.  There
are dozens more, but I would have to read more than the titles to
determine relevance, which I will do, of course, in time.

The choice facing me now is to go after \emph{depth} to solve the
COMMS/JAMMERS problem or \emph{breadth} with to understand the full
scope of the MARL field---or some combination.  Actually, maybe it is
two separate projects, so I will await direction from stakeholders.

\section{May 7, 2021}

Reading (2010) ``Particle filter based information-theoretic active
sensing,'' \cite{ryan2010particle}.  Concerns controlling robot(s)
where the control objective is to gather sensory information
(``information gain'').  This objective is described as minimizing the
entropy of an estimate distribution---the probability distribution for
the target location.  Their example is target tracking by a UAV with a
camera mounted on a fixed-wing autonomous aircraft and a moving target
on the ground.

This (2010) paper \cite{ryan2010particle} cites two more papers for
background on the methods they develop: (2004)
\cite{andrieu2004particle} and (2009) \cite{hoffmann2009mobile}.  The
first (2004) \cite{andrieu2004particle} is a review paper by leaders
of the field (that I recognized) that seems to just give an
explanation needed to understand the fundamentals of the particle
filter.  The second (2009) \cite{hoffmann2009mobile} also looks very
helpful for putting all this together.  The (2010) paper
\cite{ryan2010particle} seems to be a direct extention of the (2009)
paper \cite{hoffmann2009mobile}---the difference being that the (2010)
paper does a receding horizon control (RHC) to make it possible to
model fixed-wing aircraft that take a long time to turn as well as
mounted cameras that they claim lead to non-minimum phase behavior.
The paper \cite{ryan2010particle} uses a single aircraft as their
example (though they say their method is more general) whereas
\cite{hoffmann2009mobile} uses a swarm of maneuverable (in one time
step) UAVs.

The (2009) \cite{hoffmann2009mobile} paper cites
\cite{andrieu2004particle} as a reference for Receeding Horizon
Control (RHC).

\textbf{All three of these papers look like papers I would want to
  read closely.}

\nocite{*}

\bibliographystyle{unsrt}
\bibliography{marl}

\end{document}
