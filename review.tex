\documentclass{article} \title{Automomous Control of Aircraft for
  Communications and Electronic Warfare: The Promises of Recent
  Artifical Intelligence Literature}
\author{Sean Carver, Ph.D. at Data Machines Corporation}
\begin{document}
\maketitle
\abstract{We pose an unsolved problem in autonomous control of
  aircraft for communications and jamming (electronic warfare).  We
  aim to provide helpful pointers to literature offering approximately
  optimal solutions to related problems---such as cyber-security,
  precision farming, and search and rescue.  Solutions to these
  related problems promise applicability to diverse scenarios beyond
  the scope of the original works---including the scenario considered
  here.

  The problem we address lies within the fields of adversarial
  Multi-Agent Reinforcement Learning (MARL) and active sensing.  In
  our problem, two opposing factions (labeled ``blue'' and ``red'')
  compete to win a zero-sum/purely adversarial game.  The blue side
  tries to maintain communication links between ground-based assets
  with a fleet of ``comms'' whereas the red; whereas the red side
  tries to jam this network with a fleet of ``jammers.''  An Unmanned
  Aerial Vehicle (UAV) becomes a comm or a jammer when fitted for one
  of these purposes.

  Each faction lacks access to the state of the opposing side, and
  must infer this state probabilisticly through positioning its fleet
  for best sensory performance and localization (active sensing).
  Moreover, the ground troops of each side, positioned appropriately,
  have the possibility of shooting down any of their adversary's UAVs.
  Finally, the blue side must simultaneously achieve its objective of
  keeping units on the ground in communication.  Despite best efforts,
  different units/UAVs can fall in and out of communication with their
  respective headquarters, making each of the blue and red factions a
  multi-agent collection, fully cooperating among itself, but with
  different information, to fight its adversary having opposing goals.

  Our contribution poses this problem while pointing to literature for
  possible ideas for moving the field forward.  Finally, we pose
  certain simpler-than-reality mini-games for efficiently
  investigating solutions leading to a successful implementation for
  the full adversarial problem in real-world combat.}

\section{Introduction}
If unfortunate circumstances compel our leaders to order our armed
forces to take a city from an adversary, the command headquarters on
the ground would benefit from constant two-way communication with all
its other units during the conflict.

In the fog that accompanies such struggles, our forces cannot rely on
our enemy's network of cell towers to keep in touch.  Instead, two way
radios, linked by a network of ``comms'' will hopefully allow our
friendlies to stay connected. A comm is an unmanned aerial vehicle
(UAV) fitted for communication.

While vastly better than cell phones, such a network has it own set of
issues.  Indeed, our adversaries clearly prefer that we not speak to
each other.  To pursue this preference, they may send up jammers (UAVs
blocking communication) to keep us isolated.  Thus begins a delicate
dance of each side positioning its fleet to best find the other's
birds and in so doing best keep or block communications.

We study the question of how each side can control its fleet by
autonomously ordering and carrying out flight and communications-
electronics operation instructions (CEOI) to optimally achieve
objectives.  We are interested in the strategies for both sides,
because to defeat our enemy, we must understand the intelligent
countermeasures they may take. Moreover, in a real war, our side may
choose to put up both comms and jammers, requiring strategies for both
roles.

Much recent literature has tackled the problem of optimal search and
rescue.  This effort clearly relates to the problem at hand because,
as with rescue, each side benefits from successfully inferring the
positions of the other.  But there is a difference between search and
rescue and electronic warfare.  People being rescued presumably want
to be found and will presumably cooperate with this effort.  In
electronic warfare, it is always best if the other side cannot find
you. Thus search and rescue can succeed with a purely active sensing
and optimal control solution.  In our scenario, we need to learn to
counter our opponents strategy, whatever it may be.  To this end, we
propose to apply artificial intelligence (specifically, adversarial
multi-agent reinforcement learning).

This paper reviews the literature relevant to the comms and jammers
problem just posed.  This paper will close with a very brief
discussion presenting ideas for applying this work to other relevant
domains, notably cyber-security.

\nocite{*}

\bibliographystyle{unsrt}
\bibliography{marl}

\end{document}
