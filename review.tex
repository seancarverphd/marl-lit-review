\documentclass{article} \title{Automomous Control of Communications
  Aircraft During Combat: The Promises of Recent Artifical Intelligence
  Literature}
\author{Sean Carver, Ph.D. at Data Machines Corporation}
\begin{document}
\maketitle
\abstract{We pose a novel problem in autonomous control of
  communications aircraft and provide pointers to the literature
  offering an approximately optimal solution in many diverse
  scenarios.  The problem lies within the field adversarial
  Multi-Agent Reinforcement Learning (MARL) and Active Sensing.  Two
  factions (labeled ``blue'' and ``red'') exist within this zero-sum
  game.  The blue side tries to keep ground-based assets in
  communication with the headquarters via a fleet of comms (Unmanned
  Aerial Vehicles, or UAVs fitted for communication); whereas the red
  side tries to jam this network with a fleet of jammers (also UAV's).
  Each faction lacks full access to the state of the opposing side,
  and must infer this state through positioning its fleet for best
  sensory performance (active sensing) while simultaneously achieving
  the objective of keeping units on the ground in communication.
  Within each faction, different units/UAVs can fall out of
  communication making each of the blue and red factions a multiagent
  collection, fully cooperating among themselves but with different
  information, to fight its adversary (the other side).  Our
  contribution poses the problem while pointing to literature for
  possible ideas for implementation.}

\section{Introduction}
If unfortunate circumstances compel orders from above for our armed
forces to take a city from an adversary, the command center on the
ground benefits from constant two-way communication with all its
units.  In the fog of war, our forces cannot rely on cell phones: two
way radios, connected by a network of Comms (Unmanned Arial Vehicles
UAVs) allow friendly units to keep in touch.  But
a problem arises from the fact the fact that our adversary would
prefer that we not speak to each other, and they may send up jammers

\nocite{*}

\bibliographystyle{unsrt}
\bibliography{marl}

\end{document}
