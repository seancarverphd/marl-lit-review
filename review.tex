\documentclass{article} \title{Automomous Control of Communications
  Aircraft During Combat: The Promises of Recent Artifical Intelligence
  Literature}
\author{Sean Carver, Ph.D. at Data Machines Corporation}
\begin{document}
\maketitle
\abstract{We pose a novel problem in autonomous control of
  communications aircraft and provide pointers to the literature
  offering an approximately optimal solution in many diverse
  scenarios.  The problem lies within the field adversarial
  Multi-Agent Reinforcement Learning (MARL) and Active Sensing.  Two
  factions (labeled ``blue'' and ``red'') exist within this zero-sum
  game.  The blue side tries to keep ground-based assets in
  communication with the headquarters via a fleet of comms (unmanned
  aerial vehicles, or UAVs fitted for communication); whereas the red
  side tries to jam this network with a fleet of jammers (also UAV's).
  Each faction lacks full access to the state of the opposing side,
  and must infer this state through positioning its fleet for best
  sensory performance (active sensing) while simultaneously achieving
  the objective of keeping units on the ground in communication.
  Within each faction, different units/UAVs can fall out of
  communication making each of the blue and red factions a multiagent
  collection, fully cooperating among themselves but with different
  information, to fight its adversary (the other side).  Our
  contribution poses the problem while pointing to literature for
  possible ideas for implementation.}

\section{Introduction}
If unfortunate circumstances compel our leaders to order our armed
forces to take a city from an adversary, the command center on the
ground would benefit from constant two-way communication with all its
other units during the war.  In the fog that accompanies such
conflicts, our forces cannot rely on cell phones to keep in touch.
Instead, two way radios, connected by a network of comms (unmanned
arial vehicles, or UAVs, fitted for communication) will allow the
friendlies to stay connected.

While better than cell phones, a network of radios has it own
problems: our adversaries clearly prefer that we not speak to each
other.  To this end they may send up jammers (UAVs blocking
communication) to get in our way.

We study the question of how each side can control its fleet by
autonomously ordering and carrying out positioning and
communications-eletronics operation instructions (CEOI) to optimally
achieve objectives.  We are interested in the strategies for both
sides, because we want our systems to defeat our enemy, even if they
do something intelligent.  Moreover, in a real war, our side may
decide to put up both comms and jammers, requiring strategies for both
roles.

Much recent literature has talkled the problem of optimal search and
rescue.  This effort clearly relates to the problem at hand because
each side wants to optimally infer the positions of the other to
counter their stategy.  But there is a difference between search and
rescue and our problem.  The people being rescued presumably want to
be rescued and will cooperate with the effort, whereas our factions
presumably act to avoid being localized, with various strategies.  The
search and rescue problem allows a purely optimal control and active
sensing and control problem, we must resort to learning our opponents
strategy with artificial intelligence (adversarial multiagent
reinforcement learning).

This paper reviews the literature relevant to the comms and jammers
problem just posed.  At the end of the paper, a discussion will present ideas for applying this work to other domains, notably cyber-security.

\nocite{*}

\bibliographystyle{unsrt}
\bibliography{marl}

\end{document}
