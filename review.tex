\documentclass{article}
\title{Automomous Control of Aircraft for Communications and
  Electronic Warfare: Insights and Promises of Recent Literature}
\author{Sean Carver, Ph.D. at Data Machines Corporation}
\begin{document}
\maketitle
\abstract{We pose an unsolved problem in autonomous control of
  aircraft for communications and jamming (electronic warfare) and
  review the literature relevant to this problem. Some work offers
  approximately optimal solutions to related problems in different
  domains---promising applicability to the important scenarios
  considered here.  Other work covers methods that we may find useful
  in extending these relevant solutions.
  
  The problem we address lies within the fields of adversarial
  Multi-Agent Reinforcement Learning (MARL) and active sensing.  In
  our problem, two opposing factions (labeled ``blue'' and ``red'')
  compete to win a zero-sum/purely adversarial game.  The blue side
  tries to maintain communication links between ground-based assets
  with a fleet of ``comms;'' whereas the red side tries to jam this
  network with a fleet of ``jammers.''  An Unmanned Aerial Vehicle
  (UAV) becomes a comm or a jammer when fitted for one of these
  purposes.

  Each faction lacks knowledge and access to the state of the opposing
  side, but benefits from inferring this state probabilisticly through
  positioning its fleet for best sensory performance and localization
  (active sensing).  This maneuvering should take into account the
  real possibility of any UAV getting shot down by its adversary's
  appropriately positioned ground troops.  That said, the objective
  of each side should remain foremost.  The blue side aims to
  simultaneously keep all units in communication while the red side
  aims to simultaneously jam this communication.  Despite best
  efforts, different units/UAVs can fall in and out of communication
  with their respective headquarters, making each of the blue and red
  factions a multi-agent collection, fully cooperating among itself,
  despite different information, to fight its adversary having
  opposing goals.  Our contribution poses this problem while pointing
  to literature for possible ideas for moving the field forward
  towards a successful implementation for the full adversarial problem
  in real-world combat.}

\section{Introduction}

If unfortunate circumstances compel our leaders to order our armed
forces to take a city from an adversary, the command headquarters on
the ground would benefit from constant two-way communication with all
its other units during the conflict.

In the fog that accompanies such struggles, our forces cannot rely on
our enemy's network of cell towers to keep in touch.  Instead, two way
radios, linked by a network of ``comms'' (UAVs for communication) will
hopefully allow our friendlies to stay connected.

While vastly better than cell phones, such a network has it own set of
challenges.  Indeed, our adversaries clearly prefer to keep us out of
communication.  To pursue this preference, they may send up jammers
(UAVs for blocking communication).  Thus begins a delicate dance of
each side positioning its fleet to best find the other's birds and in
so doing best keep or block communications.

We study the question of how each side can control its fleet by
autonomously ordering and carrying out flight and communications-
electronics operation instructions (CEOI) to optimally achieve its
objectives.  We care about strategies for both sides, because to
defeat our enemy, we must understand the intelligent countermeasures
they may take. Moreover, in a real war, our side---as well as
theirs---may choose to fly both comms and jammers, requiring
strategies for both roles.

Recent literature has tackled the problem of near-optimal search and
rescue \cite{hoffmann2009mobile} and other related search and
localization paradigms \cite{ryan2010particle, tisdale2008multiple,
  gustafsson2002particle}.  Other work has considered different
applications requiring similar tools---notably cyber-security
\cite{nicholson2007information} and precision farming
\cite{testi2020reinforcement}.  Search and rescue, for example,
clearly relates to the problem at hand because, as with rescue, each
side in our conflict benefits from successfully inferring the
positions of targets on the other side.  But electronic warfare
differs from search and rescue.  People needing rescue presumably want
the search effort to succeed and will presumably not try to block
attempts to determine their positions.  In electronic warfare, on the
other hand, targets aim to conceal their true locations from their
adversaries. As a result, while search and rescue can succeed with a
purely active sensing and optimal control solution, in our scenario,
we need to learn to counter an opponent's strategy.  To this end, we
propose to apply artificial intelligence: specifically, adversarial
multi-agent reinforcement learning. This paper reviews the literature
relevant to this approach to victory.

\section{Optimal and sub-optimal filtering}
The filtering problem takes measurements of a stochastic
system---possibly transformed measurements and possibly with
noise---and produces estimates of the state of the system.  This
section reviews classical work on this problem, citing textbooks
instead of recent papers.

Readers will find the optimal solution to this problem in the first
pages of many textbooks on nonlinear filtering \cite{ristic2003beyond,
  crisan2011oxford, smith2013sequential, chopin2020introduction}.  The
solution implements a recursion consisting of alternating applications
of the Chapman-Kolmogorov Equation and Bayes Rule.

Unfortunately, solving each of these equations demands an integration
remaining provably intractable in most cases---indeed in all but two
cases that researches have already identified.  In all other cases, a
researcher must settle for an approximation---a sub-optimal (but
hopefully still \emph{approximately} optimal) filter.  Readers will
find that the rest of the nonlinear filtering textbooks (the rest
beyond the first few pages devoted to the optimal exposition) develop
these sub-optimal approximations.

First, we list the two truly optimal solutions to the filtering
equation as (1) the Kalman filter, and (2) the finite hidden Markov
filter.  The Kalman filter uses a linear model of the process, a
quadratic objective function measuring optimality, and Gaussian noise
corruption (LQG problem).  The Gaussian assumption must hold for both
the process noise and the measurement noise.  The LQG problem divides
into the linear quadratic estimator (LQE) problem for the optimal
state estimates and the linear quadratic regulator (LQR) for the
optimal control of such a system.  As we will see below, the solutions
of these problems decouple in a surprising and useful way (the
separation principle, see the last section below) but unfortunately,
this decoupling makes use of the specific assumptions we have imposed
here.

Likewise, a finite hidden Markov filter uses a finite-state model of
the process. These restrictions unacceptably constrain usable models
for our application area, and therefore we will focus on approximately
optimal alternatives.

Several specific approximations merit mention.  An extended Kalman
filter linearizes the state space around each sample point allowing
the calculations behind the Kalman filter to proceed approximately,
even when the assumptions for using a Kalman filter do not hold
exactly. The approximation works well when the optimal probability
distributions for state remain close to Gaussian.  If they do not
remain approximately Gaussian, the Extended Kalman Filter can perform
poorly, leading to poor state estimates and impoverished inference.

A second approximation, a grid-based filter, approximates the state
space with a finite grid of points allowing the calculations behind a
hidden (finite) Markov filter to proceed, even for infinite state
spaces.  A grid-based filter works well for the lowest dimensional
state spaces, but becomes computationally intractable when the
dimensions become even slightly higher.  In preliminary investigations
the electronic warfare problem that motivated this review [Carver,
research paper in preparation], one and two targets worked well (each
adding two dimensions, longitude and latitude, to the state), whereas
three simultaneous targets remained expensive beyond reach.  In this
work, we aimed to find jammers (``targets'') without bearing
information from observing successful or unsuccessful radio
connections to friendlies.

The field calls the last class of filtering approximations that
deserves our attention "particle filters."  In short, the idea
approximates evolving distributions with a finite swarm of Monte Carlo
sample points called particles.  These methods possess great
generality and flexibility, but many researchers find particle methods
more difficult to understand, and to successfully implement, than
their simpler and more straightforward cousins.  Note that there exist
many different ways to implement particle filters, each with its own
benefits and limitations.  We will discuss these methods further in
the next section, as several papers concerning Active Sensing use
particle filters.

\section{Active sensing}

Active sensing solves a control problem, and as such has considerable
overlap with reinforcement learning.  Both domains use observations to
select actions with some notion of how to make that selection.
Whereas reinforcement learning tries to optimize cumulative reward,
many implementations of active sensing choose its action (they call it
control) to maximize information or, equivalently, minimize entropy,
in the distributions for the estimated quantities.  For example, in
active sensing for search and rescue, we want to control the sensors
(choose the action) to best reduce the uncertainty in the target
locations.

The electronic warfare application that we consider does not have
exactly this form.  We aim not just to locate the other side's birds,
but also we aim to keep (or for red, block) communication between blue
units.  This additional objective lands the problem squarely in the
purvue of reinforcement learning.  That said, the objective of
electronic warfare benefits from knowing, even with uncertainty, the
locations of the other side's UAVs.  Moreover, like with active
sensing, each side can move its own fleet to localize the other's, but,
with electronic warfare, it becomes a means to the end of meeting its
other objectives.  Therefore, we look to the literature on active
sensing only for inspriation.

We start with one representative paper in this field
\cite{hoffmann2009mobile}, by Hoffman and coauthors.  Hoffmann et al.\
considers the problem of target localization, with a search and rescue
application in mind.  They consider a number of mobile sensor vehicles
(analogous to blue's comms) and a number of fixed-location targets
(analogous to the red's jammers that blue tries to find by moving its
comms).  While Hoffmann uses fixed targets, he cites
\cite{gustafsson2002particle} for a motion model that would
generalizes his solution to moving targets.  Both of these papers
advance particle filter solutions.  Hoffmann specifically closes the
loop with a control that minimizes the uncertainty in the target
locations.  In that sense, Hoffmann implements an active sensing
paradigm.

It becomes interesting to consider the consequences of the change in
objective function needed for our application.  For search and rescue,
the mobile sensors care only about the uncertainty in the state
distribution of the targets to move its fleet to localize the targets.
Our application changes this picture: if there exist directions of
variation that do not matter as much for the objective of keeping
targets in communication, our agents may care less about the
uncertainty in these directions.

We propose to codify the comms and jammers objective function by
giving a reward to each agent based on which units they succeed in
directly or indirectly (eg.\ through secondary connections)
contacting.  Each agent then tries to maximize the cumululative
discounted expected value of this reward.

Hoffmann considers two cases separately: bearings-only measurements,
and range-only meausurements.  Both will become useful simplifications
in our case.  Taking inspiration from Hoffmann, we can test the
performance of our agents under each of these constraints (ie.\
bearings-only and range-only measurements) separately, then together
in both homogeneous and heterogeneous mixtures of constraints across
sensors.

Extending the work of Hoffmann, Ryan et al. \cite{ryan2010particle}
use non-trivial models for the sensor dynamics appropriate for
fixed-wing UAVs that may tens of seconds to execute a maneuver, such
as a 180 degree turn.  In constrast, the quadrotor robots
\cite{hoffmann2007quadrotor} of our models have simple dynamics and
can essentially execute maneuvers in a single time step.  While our
work will initially consider UAVs of the sort that Hoffmann considers,
Ryan's paper generalizes the problem statement in an important
direction in that it removes an assumption about the sensor platforms
that does hold for the quadrorotor devices we model here, but fails
for other UAVs, such as fixed-wing aircraft.

Another paper \cite{tisdale2008multiple}, by some of the same authors,
presents the related particle filter-based unifying approach to search
and tracking.


\section{Reinforcement learning and extensions}

This section spans several disciplines, including reinforcement
learning, deep reinforcement learning, distributional reinforcement
learning, Bayesian reinforcement learning, and multi-agent
reinforcement learning (see citations below).

We start each topic by defining the relevant terms listed above, and
providing one or more citations.  Reinforcement learning (RL)
\cite{sutton2018reinforcement, kaelbling1996reinforcement} extends
machine learning to sequential problems where an agent or agents learn
to interact with an environment to maximize cumulative reward.

Researchers have solved similar problems in different disciplines.
Indeed, RL shares a sizable overlap with control theory in
engineering, including active sensing.  In engineering, adaptive
control \cite{aastrom2013adaptive, khan2012reinforcement} lies closest
to RL.  The terminology of engineering differs from the terminology of
reinforcement learning, but the terms map perfectly to one another.
The ``controller'' (agent) interacts with the ``plant'' or ``process''
(the environment) by selecting a ``control'' (action) that ``minimizes
cost'' (maximizes reward).

Engineers typically deal with continuous systems (ie robotics),
whereas many, but not all, reinforcement learning solutions have
finite time steps, finite action spaces, as well as, often, finite
state spaces.  The treaatment remains essentially the same, but may
look a little different.  For example, in engineering an analogous
integral of cost can replace the RL return consisting of a discounted
accumulation of rewards.

Engineers have fully developed the linear theory, which reinforcement
learning has barely touched.  In another difference, adaptive control,
often aims more to maintain control of the plant in the face of
natural but unpredictable changes, rather than, as aimed for often in
RL, to learn to control the plant \emph{de novo}.

Despite cosmetic differences, if both systems solve the same problem,
and to the extent that they solve it optimally---with the same
optimality criteria---the solutions must tautologically coincide.  But
the approximations to optimality made in each discipline may differ,
and the assumptions and problem formulations can also differ, as well.
Undoubtedly, we would find a similar truth to analogous statements
made of connections betweeen RL and active sensing.

Deep RL \cite{li2017deep} uses neural networks to represent the
functions learned by the agent(s).  Classically, RL implementations,
including many deep RL implementations, deal with inevitable
uncertainty in represented quantities by maintaining best point
estimates for these quantities.  Distributional RL
\cite{osband2018randomized} departs from this tradition by maintaining
full probability distributions for the uncertain quantities. If the
actor(s) also perform Bayesian inference on these distributions (as
they generally do), the agent implements Bayesian RL
\cite{ghavamzadeh2016bayesian}.

As an approach to our application, we would like to direct the
reader's attention to distributional RL, its slightly smaller subset,
Bayesian RL.  Consider that the target localization and active sensing
literature of interest represent uncertain quantities with
distributions, just like distributional RL.  While an RL algorithm
could derive point state estimates from distributional state
information to decide on an action, such an approach seems wasteful.
Achieving an optimal solution to the electronic warfare problem stands
as a worthy, if impossible, ambition for a machine learning engineer.
That said, to not use the full distribution returned by filtering
amounts to giving up on this ambition, or so it seems.

However, the mathematics can sometimes work out so that throwing away
the distribution for a point estimate succeeds as the optimal solution
\cite{aastrom2012introduction}.  For example, to control a linear
regulator, where a quadratic cost function determines optimality,
corrupted by Gaussian noise, (the LQR problem) the optimal solution
uses a Kalman filter to produce the optimal state (point) estimate as
a function of time.  Then the solution applies the optimal control for
a determinisitic regulator given that the ``assumed known'' state
equals the state (point) estimate.  In this case, throwing away the
distribution allows a simpler, more parisimonious, and indeed still
fully optimal solution.

A system satisfies the so-called \emph{separation principle} if such a
separation between estimation and control holds for the system.  But
systems do not always satisfy the separation principle---those that do
stand as the exceptions.  If an engineer can get away with invoking
the separation principle, the calculations greatly simplify.  Indeed,
for the LQR problem, the solution exists in closed form.

We do not know how well the separation principle applies in our
problem and we do not know how computationally expensive it will
become to use purely distributional methods.  The approach we suggest
applies distributional RL (or, more precisely, Bayesian RL) to
simplified versions of the problems first, then push the envelope both
in terms of the complexity of the scenario, and in terms of the
approximations used---such as the separation principle hoping that it
applies at least approximately.

In the LQR problem we have a precise notion of optimality (the
quadratic cost function) and a proof that the solution presented
optimizes this condition.  In our comms and jammers scenario we have
neither.  Moreover, a lengthy (and uncertain to succeed) search for
such a proof may have little to no value for our clients.

We could say, however, that we will apply principles, such as Bayes
Rule, that researchers in other contexts have shown lead to provably
optimal solutions.  In that sense we could say that ``optimal''
principles underlie our methods, rather than that our solutions
achieve optimality.  Approaching the problem in this way, we hope that
our solutions will stand as very good, indeed good enough, and perhaps
even the very best---unproved, of course.

Finally multi-agent RL \cite{bucsoniu2010multi} extends RL to
environments that include other interacting agents cooperating or
competing for reward.  The definition of ``multi-agent'' given the
introduction to Shoham's textbook \cite{shoham2008multiagent} requires
that the agents have different information, different interests
(quantified as reward), or both.

By this definition, Hoffmann's implementation of search and rescue
\cite{hoffmann2009mobile}, discussed above, does \emph{not} qualify as
a multi-agent system because (1) all agents share all relevant
information over a radio channel, and (2) they all have the same
interests in localizing the target(s).  This conclusion holds even
though each sensor platform behaves independently of the others and
performs its own calculations (which, because of shared information,
the authors expect to remain identical to the calculations of the
other agents, as the search proceeds).

On the other hand, our comms and jammers implementation does land
squarely in the multi-agent camp because the blue and red sides
certainly do have different interests, and moreover, barring unmodeled
espionage, certainly do have different information.  The two sides
have not just different interests, but diametrically opposed
interests.

To reiterate, each faction has its own sensors and uses measurements
which it hides from its adversaries.  But even within each faction,
some units can access information that other units cannot.  Indeed,
the individual blue and red units may yet find themselves out of
communication with their compatriots.  Thus, even assuming factions
share interests, they may not always share all information.  As a
result, there can exist more than two agents (by Shoham's definition)
in our environment.  These realities become challenging wrinkles which
we must deal with for success.

Reinforcement learning algorithms divide into \emph{model-based}
algorithms and \emph{model-free} algorithms.  The two methods differ
in that a model-based algorithm maintains a model of the environment
and uses this model to select its actions.

Model-based reinforcement learning would fail catastrophically with
our scenario in the real world.  The model in the model-based
algorithm would have to include models of all other agents, including
those of our adversary.  But we must expect that our adversaries will
exploit any modeling assumptions we make---and posing a model of our
adversaries requires such assumptions.

The model-free alternative presents itself as an option.  Within
Bayesian reinforcement learning, a recent comprehensive survey of BRL
\cite{ghavamzadeh2016bayesian} discusses and develops only two such
classes of algorithms in that context: \emph{Bayesian Policy Gradient
  Algorithms} \cite{engel2007bayesian, ghavamzadeh2016abayesian} and
\emph{Bayesian Actor-Critic Algorithms} \cite{ghavamzadeh2007bayesian,
  ghavamzadeh2016abayesian}.  Of these two possibilities the authors
of \cite{ghavamzadeh2016bayesian} caution that their Bayesian actor
critic implementation takes advantage of the Markov property, but that
the Bayesian policy gradient algorithm does not make this assumption
and stands appropriate for partially observed problems including
Markov games.

These constraints, together with an initial preference for BRL,
narrow, to a single option, the initially wide field of RL algorithms
of interest along this path.  However, there remain other trade offs
to consider, so we should not focus on purely Bayesian methods.
Indeed, it seems that just one set of authors has produced most of the
work on model-free Bayesian RL, with the rest of the community slow to
adopt these (admittedly somewhat recent) ideas.  Moreover while these
authors discuss applying their methods in a multi-agent framework,
they intend their work mostly for single- or few-agents.  They concede
that their methods require a posterior over all policies of all
agents, which may remain tractable in some cases, such as agents
having independent policies, but this simplification seems
questionable for our application.  The authors also mention the
possibility of addressing this constraint with a myopic (1-step) look
ahead on the value function and cite \cite{dearden1998bayesian}, an
article about Q-learning.  But it remains unclear to what extent BRL
will have comparable capabilities to the capabilities of the methods
intended for and tested on primarily multi-agent environments.  Never
the less, we deem that these techniques still merit a close look and
careful understanding.

Both Bayesian Policy Gradients and the Bayesian Actor Critic model the
time varying gradient of the objective function used to train the
policy as a Gaussian process (GP)---a type of stochastic process on
the real line or, in our case, a discrete subset where we have data.
This objective computes the expected return (discounted sum of all
rewards).  A GP assumes that all its finite dimensional distributions
(ie.\ the joint distribution of a finite sample of the domain) have a
multidimensional Gaussian distribution. With GPs, Bayesian policy
gradients (and Baysian actor-critic) can perform inference on the
posterior of the gradient conditioned on the data, which allows for
the success of the method.  Still it remains unclear to what extent
the Gaussian process assumption has limitations.

For this reasons, together with the lack of testing on multi-agent
systems, we cast a wider net by considering recent work of several
authors concerning other techniques for multi-agent RL, particualarly
those specifically intended for adversarial applications.

In a recent paper, Lowe et al. \cite{lowe2017multi} considers an
actor-critic paradigm, but one that imposes centralized training to
prepare for decentralized execution.  Lowe and coauthors cite a recent
body of previous work briefly reviewed in their paper.  In this
paradigm, for all agents, there exists one single central critic
trained with the all agents observations (or the whole state), and all
actions (or all policies) of all agents.  After training, during
testing and execution no centralizations remain, and each agent
executes its own decentralized but pre-trained critic.  On the other
hand, each agent's actors in this paradigm remains always
decentralized.  Though the actors and never get the benefit of
centralized training, the success of Lowe's methods apparently depends
upon consistently decentralized actors.

A second paper by Srinivasan et al., \cite{srinivasan2018actor},
considers an alternative approach which never centralizes the
training.  Decentralized training, as with decentralized testing and
execution, may matter to us if we want to train our agents as they
perform in the field.  Of course this property matters most in
situations in which we must adapt and counter, on the fly, to our
adversaries adapting and countering us.  Indeed Scrinivasan and
coauthors intend their work for adversarial

A different paper, \cite{su2020counterfactual}, considers multi-agent
actor-critics in situations where different agents cooperate to
maintain a graph of two-way connections---a relevant capability for
comms and jammers within each faction, though, in this paper, without
the spector of competition from the other side.

Finally Foerster et al....

% \nocite{*}
\bibliographystyle{ieeetr}
% \bibliographystyle{unsrt}
\bibliography{marl}

\end{document}
