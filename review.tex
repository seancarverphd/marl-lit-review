\documentclass{article} \title{Automomous Control of Aircraft for
  Communications and Electronic Warfare: The Promises of Recent
  Artifical Intelligence Literature}
\author{Sean Carver, Ph.D. at Data Machines Corporation}
\begin{document}
\maketitle
\abstract{We pose an unsolved problem in autonomous control of
  aircraft for communications and jamming (electronic warfare) and
  review the literature relevant to this problem. Some work
  offers approximately optimal solutions to related problems in
  different domains---promising applicability to the
  important scenarios considered here.  Other work covers methods that
  we may find useful in extending these relevant solutions.
  
  The problem we address lies within the fields of adversarial
  Multi-Agent Reinforcement Learning (MARL) and active sensing.  In
  our problem, two opposing factions (labeled ``blue'' and ``red'')
  compete to win a zero-sum/purely adversarial game.  The blue side
  tries to maintain communication links between ground-based assets
  with a fleet of ``comms;'' whereas the red side tries to jam this
  network with a fleet of ``jammers.''  An Unmanned Aerial Vehicle
  (UAV) becomes a comm or a jammer when fitted for one of these
  purposes.

  Each faction lacks access to the state of the opposing side, and
  must infer this state probabilisticly through positioning its fleet
  for best sensory performance and localization (active sensing).
  Moreover, the ground troops of each side, when also positioned
  appropriately, have the possibility of shooting down any of their
  adversary's UAVs.  The blue side must simultaneously
  achieve its objective of keeping units on the ground in
  communication and the red side must simultaneously try to jam this
  communication.  Despite best efforts, different units/UAVs can fall
  in and out of communication with their respective headquarters,
  making each of the blue and red factions a multi-agent collection,
  fully cooperating among itself, but with different information, to
  fight its adversary having opposing goals.

  Our contribution poses this problem while pointing to literature for
  possible ideas for moving the field forward.  We discuss the use of a
  heirarchy of simpler-than-reality mini-games for efficiently
  investigating and building upon solutions leading to a successful
  implementation for the full adversarial problem in real-world combat.}

\section{Introduction}

If unfortunate circumstances compel our leaders to order our armed
forces to take a city from an adversary, the command headquarters on
the ground would benefit from constant two-way communication with all
its other units during the conflict.

In the fog that accompanies such struggles, our forces cannot rely on
our enemy's network of cell towers to keep in touch.  Instead, two way
radios, linked by a network of ``comms'' (UAVs for communication) will
hopefully allow our friendlies to stay connected.

While vastly better than cell phones, such a network has it own set of
challenges.  Indeed, our adversaries clearly prefer to keep us out of
communication.  To pursue this preference, they may send up jammers
(UAVs for blocking communication).  Thus begins a delicate dance of
each side positioning its fleet to best find the other's birds and in
so doing best keep or block communications.

We study the question of how each side can control its fleet by
autonomously ordering and carrying out flight and communications-
electronics operation instructions (CEOI) to optimally achieve its
objectives.  We are interested in the strategies for both sides,
because to defeat our enemy, we must understand the intelligent
countermeasures they may take. Moreover, in a real war, our side---as
well as theirs---may
choose to fly both comms and jammers, requiring strategies for both
roles.

Much recent literature has tackled the problem of optimal search and
rescue.  Other work has considered different scenarios requiring similar 
tools---notably cyber-security and precision farming.  CITE 3.
Search and rescue clearly relates to the problem
at hand because, as with rescue, each of our sides benefits from
successfully inferring the positions of targets.  But there is a
difference between search and rescue and electronic warfare.  People
being rescued presumably want to be found and will presumably cooperate 
with this effort.  In electronic warfare, on the other hand, participants
always aim to keep their locations hidden from the other side. As a result,
while search and rescue can succeed with a purely active sensing and
optimal control solution, in our scenario, we need to learn to counter
an opponent's strategy.  To this end, we propose to apply artificial
intelligence: specifically, adversarial multi-agent reinforcement
learning. This paper reviews the literature relevant to this approach to
electronic warfare.

\section{A Heirarchy of Models}

We propose models of the battlefield together with capabilties of each
side.  However, we aim to make our method successful against a
real-world adversary on a real-world battlefield, not just against one
particular model adversary on one particular model battlefield.  The
approach we suggest implements a heirarchy of models, from trivial to
complex.  For the purposes of the exposition, we illustrate just a few
pieces of this heirarchy.

We [at Data Machines Corporation (DMC)] previously supported a
simulation environment (ULTRA) to train human commanders to take a
city from an adversary.  The simulation platform (developed outside of
DMC) showed stunning complexity and realism in its details.  It
included electronic warfare as well as a large array of other features
of the conflict.

One might conceive of the following challenge in Artificial
Intelligence: consistently beat at ULTRA the best human players of
ULTRA.  In this sense, ULTRA finds an analog with the dizzyingly
complex game StarCraft II, cracked several years ago by reseachers at
DeepMind \cite{vinyals2019grandmaster}.  That said, for electronic
warfare, we care about something harder: we care not just that we can
beat the best human opponents in the game, but more importantly we
care that we can beat our future adversaries on real-world
battlefields.

A bot that excels against human players of ULTRA has merit as a goal,
but we must worry that that in training against ULTRA will train for
any peculiarities of the game that do not exist in the real world---if
any such peculiarities actually exist in the game.  To counter this
possibility, our approach trains our bots against a wide variety of
scenarios---both complex scenarios within ULTRA, and simple, even
trivial, ones outside of ULTRA.  Simple and trivial scenarios, while
in themselves will not win a war, can lead to insight about what
happens in training, which makes a step in that direction.

\section{Even trivial scenarios reveal complexities}

\section{A moderately complex model}
The field of interest is a square in the Euclidean plane.  In this
field, there is a ``source,'' or ``sender,'' labeled $S$, a receiver,
labeled $R$, and $n$ different ``jammers,'' labeled $J_1, \dots, J_n$.
We do not assume that the ability to communicate is symmetric---a
separate calculation is needed when the roles of sender and receiver
are reversed.  Additionally there are $m$ different comms to
facilitate communication.

\section{Optimal Placement Of Comms}

Let $c_i$ denote positions of comms and let $j_k$ denote positions
position of jammers.  If these quantities are known, we can compute
the probability of a successful communication between the headquarters
and an asset (see more details below).  We denote this probability

$$P(\mbox{transmission}|c_i,j_k)$$

Some choices still need to be made about how this probability will be
computed, but it shouldn't be hard.  By total probability,

$$P(\mbox{transmission}|c_i) = \sum_k P(\mbox{transmission}|c_i,j_k)
  P(j_k)$$

It is a sum rather than an integral because we only consider grid
points as possible locations.  That may need to be generalized.  The
second factor of each term in the sum is the posterior probability
already coded for computation and described below.

A final step in the computation is a optimization over the comm
positions $c_i$.

\section{TL;DR}
Let $P_{j}(d_j)$ be the power of jammer $j$ at the receiver, at
distance $d_j$.  Let $P_S(d_S)$ be power of the source at the
receiver, at distance $d_S$.  Let $P_N$ be the power of the ambient
noise.

These functions are given by $P_j = \frac{M_j}{d_j^2}$, for $j = 1,
\dots, n$ and $P_S = \frac{M_S}{d_S^2}$; $P_N$ is a constant.

The probability of a successful transmission is given by

$$\mbox{sigmoid}\left( 10 \log_{10}\left(\frac{P_S(d_S)}{P_1(d_1) +
  P_2(d_2) + \dots + P_n(d_n) + P_N} \right) \right)$$

The quantity in the sigmoid is the strength expressed in decibels, of
the signal against the background that includes all jammers and the
ambient noise, see Equation (1) in reference \cite{parlin2018jamming}

\section{Some thoughts about the man-in-the-middle}
Billy had alerted us to the fact that it was more than just distance
to jammer--jammers in the line of sight were more potent than those
off of it.

I spent a long time trying to justify putting this feature explicitly
in the model.  But the model above seemed most consistent with what I
was reading in the literature and it made sense.  It is distance to
receiver that matters.  A man in the middle generally will jam the
receiver, unless distances are great enough, the ambient noise is
small enough, and the proportionality constant of the jammer is small
enough compared to the proportionality constant of the sender.  This
makes sense to me.

There is such a thing as electomagnetic interferece.  According to
Wikipedia: ``The effect of unwanted energy due to one or a combination
of emissions, radiations, or inductions upon reception in a
radiocommunication system, manifested by any performance degradation,
misinterpretation, or loss of information which could be extracted in
the absence of such unwanted energy.'' My reading of this is this:
interference happens upon the reception of the signal, in the circuits
of a radiocommunication system, not with signal itself.
Electomagnetic waves don't interact, as far as I am aware, but it can
be impossible to pick out the signal in the noise.

Having said all that, a directional receiving antenna would sharpen
the effect of a man-in-the-middle jammer, versus one off to the side,
but for the moment I'm keeping it simple and assuming the antennae
have no preferred direction.  Could change that next, though, by
assuming they are optimally directed to receive the signal in a
particular orientation, and to reject signals in substantially
different directions.

\section{Power}

The power of the radio signal from a point source at a distance $r$ is
given by (assuming the free-space propagation loss model):
$$P_{\mbox{signal}}(d) = \frac{M}{d^2},$$ where $M$ is a
proportionality constant that depends on units, and even with the same
units, can be different for different senders.  Note that $M$ is the
power of the signal at distance 1.  The value of $M$ may be know for
friendlies but need to be extimated for jammers.  Equation (1) in
reference \cite{benner1996effects}, showed a similar equation for
``path loss'' relaxing the free-space propagation loss model with a
proportionality exponent possibly different from 2.  It seems that the
units of distance are important in this model---see reference
\cite{benner1996effects}. I note this observation for the future as I
intend to use the free-space propagation loss model for now.


Equation (5) in reference \cite{xu2007adjusting} gives
$$M = \frac{P_T G_T G_R}{4 \pi},$$ where $P_T$ is the power of the
transmitter, $G_T$ is the gain of the transmitter in the direction of
the receiver, and $G_R$ is the gain of the receiver in the direction
of the transmitter.  Presumably the transmitter could be either the
jammer or the source.  Reference \cite{benner1996effects}, gives a
similar equation that may not be entirely consistent that has a
dependence on wavelength, Equation (1).  The basic thing I get from
this is kind of obvious for anyone raised in the era before cable
television---it depends on the orientation of the anntena how strong
the signal.  Again, I am just putting this observation in this
document for the future.  For the first cut the power received by the
anntena is just a function of the distance to the transmitter, not the
orientation of the antenna.

\section{Jamming}

One reference I have \cite{xu2007adjusting} talks about channel
capacity as the maximum bit rate $S \rightarrow R$, under the
assumption of error correcting codes, etc.  So it seems in this
paradigm that it is not whether you are jammed or not, but how many
bits get through per second.  In several references, including this
one there is a noise term which adds power to the jammer in a
symmetric way.  When the decibels in the equation above drops below
the 0, the channel capacity falls below the bandwidth of the channel.

I am going to ignore the details of channel capacity and error
correcting codes and just say probability of transmission is a
function of the ratio of the powers of signal to background, expressed
as a sigmoid of decibels.

\nocite{*}
\bibliographystyle{ieeetr}
% \bibliographystyle{unsrt}
\bibliography{marl}

\end{document}
